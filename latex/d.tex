\documentclass[12pt]{article}

\usepackage[a4paper,margin=1in]{geometry}
\usepackage{amsmath,amssymb}
\usepackage{enumitem}
\usepackage{multicol}

\setlist[enumerate]{leftmargin=*, itemsep=0.75em}

\begin{document}

\begin{center}
{\Large \textbf{A Math Quadratic - Consolidated Questions - Chp 1}}\\[6pt]
\end{center}

\noindent\textbf{Instructions.} Answer all questions. Where a sketch is requested, sketch the graph clearly and label all key features (turning point, intercepts, and line of symmetry where appropriate).

\vspace{1em}

\begin{enumerate}[label=\textbf{\arabic*.}]

\item Express $2x^2-7x+1$ in the form $a(x+b)^2+c$.

\item Express $-4x^2+12x-5$ in the form $a(x+b)^2+c$.

\item Express $\dfrac{3}{2}x^2-5x+\dfrac{7}{3}$ in the form $a(x+b)^2+c$. Give exact values.

\item Express $x^2-\dfrac{7}{3}x+\dfrac{5}{9}$ in the form $(x+a)^2+b$.

\item A quadratic curve has turning point $(-3,5)$ and passes through the point $(1,-3)$.
\begin{enumerate}[label=(\roman*)]
\item Find the equation of the curve in the form $y=a(x-h)^2+k$.
\item Hence write the equation in the form $y=ax^2+bx+c$.
\end{enumerate}

\item For $y=4x^2-12x+5$:
\begin{enumerate}[label=(\roman*)]
\item Express $y$ in the form $a(x-h)^2+k$.
\item Sketch the graph and state the turning point, the $x$-intercepts (if any), the $y$-intercept, and the line of symmetry.
\end{enumerate}

\item For $y=-2x^2-4x+3$:
\begin{enumerate}[label=(\roman*)]
\item Express $y$ in the form $a(x-h)^2+k$.
\item Sketch the graph and state the turning point, the $x$-intercepts (if any), the $y$-intercept, and the line of symmetry.
\end{enumerate}

\item Sketch the graph of $y=(x-4)(x+2)$ and state the $x$-intercepts, the $y$-intercept, the turning point, and the line of symmetry.

\item A quadratic curve is U-shaped, has $x$-intercepts $-1$ and $5$, and passes through $(0,-10)$.
\begin{enumerate}[label=(\roman*)]
\item Find the equation of the curve in expanded form.
\item State the turning point and the line of symmetry.
\item Sketch the graph.
\end{enumerate}

\item A quadratic curve opens downward, has line of symmetry $x=1$, and passes through $(1,4)$ and $(3,0)$.
\begin{enumerate}[label=(\roman*)]
\item Find the equation of the curve.
\item Find the intercepts and state the turning point.
\item Sketch the graph.
\end{enumerate}

\item Let $f(x)=-2x^2+10x-7$.
\begin{enumerate}[label=(\roman*)]
\item Express $f(x)$ in the form $a(x-h)^2+k$.
\item Hence state the maximum value of $f(x)$ and the value of $x$ at which it occurs.
\item Solve the equation $f(x)=1$.
\end{enumerate}

\item For $y=x^2-6x+13$:
\begin{enumerate}[label=(\roman*)]
\item Find the minimum value of $y$ and the value of $x$ at which it occurs.
\item Find the values of $x$ for which $y=17$.
\end{enumerate}

\item The quadratic function is $y=-x^2+ax+4$, where $a$ is a constant. The maximum value of $y$ is $9$.
\begin{enumerate}[label=(\roman*)]
\item Find the possible values of $a$.
\item For each value of $a$, state the turning point.
\end{enumerate}

\item The graph of $y=2x^2+px+q$ has line of symmetry $x=3$ and passes through the point $(0,6)$.
\begin{enumerate}[label=(\roman*)]
\item Find $p$ and $q$.
\item Find the turning point and the $x$-intercepts.
\item Sketch the graph.
\end{enumerate}

\item The equation $x^2-(p+1)x+p=0$ has two real roots that differ by $3$. Find $p$.

\item A firework leaves the ground at $x=1$ m and lands at $x=9$ m. Its maximum height is $8$ m at $x=5$ m.
\begin{enumerate}[label=(\roman*)]
\item Find a quadratic model for its height $y$ (m) in terms of $x$ (m).
\item Find $y$ at $x=7$.
\item A wall at $x=7$ is $6$ m tall. Determine whether the firework clears the wall.
\item Find the values of $x$ for which $y=6$.
\item Sketch the graph.
\end{enumerate}

\item A suspension cable is modelled by
\[
y=\frac{1}{500}(x-100)^2+6,\qquad 0\le x\le 200,
\]
where $x$ (m) is the horizontal distance from the left tower and $y$ (m) is the height above the road.
\begin{enumerate}[label=(\roman*)]
\item Find the height of each tower above the road.
\item Find the height of the cable at $x=40$.
\item A truck is $12$ m tall. Find the values of $x$ at which the cable height equals the truck height (i.e.\ the boundary points where the truck can just pass).
\item Sketch the graph.
\end{enumerate}

\item A farmer has $200$ m of fencing to enclose three identical rectangular pens side-by-side against a straight river (no fencing along the river). Let $x$ m be the dimension perpendicular to the river and $L$ m be the total length along the river.
\begin{enumerate}[label=(\roman*)]
\item Express the total area $A$ in terms of $x$.
\item Find the maximum total area and the corresponding values of $x$ and $L$.
\end{enumerate}

\item Monthly profit (in dollars) from producing $x$ hundreds of units is
\[
P(x)=-2x^2+180x-2000.
\]
\begin{enumerate}[label=(\roman*)]
\item Find the value of $x$ that gives the maximum profit and the maximum profit.
\item Find the break-even values of $x$.
\item Find the values of $x$ for which $P(x)=1600$.
\end{enumerate}

\item A performer's height $h$ (m) after travelling $x$ m horizontally is quadratic. He starts at height $2$ m when $x=0$, passes through $(10,12)$, and lands on the ground (i.e.\ $h=0$) at $x=30$.
\begin{enumerate}[label=(\roman*)]
\item Find $h(x)=ax^2+bx+c$.
\item Find the maximum height and the value of $x$ where it occurs.
\item A barrier at $x=12$ is $13$ m high. Determine whether he clears the barrier.
\item Sketch the graph.
\end{enumerate}

\end{enumerate}

\newpage
\section*{Solutions}
\begin{multicols}{2}
\footnotesize

\begin{enumerate}[label=\textbf{\arabic*.}, leftmargin=*]

\item $2x^2-7x+1=2\left(x-\dfrac{7}{4}\right)^2-\dfrac{41}{8}$.

\item $-4x^2+12x-5=-4\left(x-\dfrac{3}{2}\right)^2+4$.

\item $\dfrac{3}{2}x^2-5x+\dfrac{7}{3}=\dfrac{3}{2}\left(x-\dfrac{5}{3}\right)^2-\dfrac{11}{6}$.

\item $x^2-\dfrac{7}{3}x+\dfrac{5}{9}=\left(x-\dfrac{7}{6}\right)^2-\dfrac{29}{36}$.

\item Turning point $(-3,5)$ gives $y=a(x+3)^2+5$.
Using $(1,-3)$: $-3=16a+5\Rightarrow a=-\dfrac12$.
So $y=-\dfrac12(x+3)^2+5$.
Expanding: $y=-\dfrac12x^2-3x+\dfrac12$.

\item $y=4\left(x-\dfrac{3}{2}\right)^2-4$.
Turning point $\left(\dfrac{3}{2},-4\right)$, axis $x=\dfrac{3}{2}$,
$y$-intercept $(0,5)$, $x$-intercepts $x=\dfrac12,\dfrac52$.

\item $y=-2(x+1)^2+5$.
Turning point $(-1,5)$, axis $x=-1$,
$y$-intercept $(0,3)$,
$x$-intercepts $x=-1\pm\dfrac{\sqrt{10}}{2}$.

\item $x$-intercepts $(-2,0),(4,0)$; $y$-intercept $(0,-8)$;
axis $x=1$; turning point $(1,-9)$.

\item $y=a(x+1)(x-5)$ and $y(0)=-10$ gives $-5a=-10\Rightarrow a=2$.
So $y=2(x+1)(x-5)=2x^2-8x-10$.
Axis $x=2$, turning point $(2,-18)$.

\item Vertex at $(1,4)$ gives $y=a(x-1)^2+4$ and $y(3)=0$ gives $4a+4=0\Rightarrow a=-1$.
So $y=-(x-1)^2+4=-x^2+2x+3$.
$x$-intercepts $x=-1,3$; $y$-intercept $(0,3)$; turning point $(1,4)$.

\item $f(x)=-2\left(x-\dfrac{5}{2}\right)^2+\dfrac{11}{2}$.
Maximum value $\dfrac{11}{2}$ at $x=\dfrac{5}{2}$.
Solve $f(x)=1$:
$-2\left(x-\dfrac{5}{2}\right)^2+\dfrac{11}{2}=1
\Rightarrow \left(x-\dfrac{5}{2}\right)^2=\dfrac{9}{4}
\Rightarrow x=1 \text{ or } 4$.

\item $y=(x-3)^2+4$.
Minimum value $4$ at $x=3$.
For $y=17$: $(x-3)^2=13\Rightarrow x=3\pm\sqrt{13}$.

\item $y=-\left(x-\dfrac{a}{2}\right)^2+\dfrac{a^2}{4}+4$.
Maximum value is $\dfrac{a^2}{4}+4=9\Rightarrow a^2=20$.
So $a=\pm 2\sqrt5$.
Turning point is $\left(\dfrac{a}{2},9\right)$, i.e.\ $(\sqrt5,9)$ or $(-\sqrt5,9)$.

\item Axis $x=3$ gives $-\dfrac{p}{4}=3\Rightarrow p=-12$.
Using $(0,6)$ gives $q=6$.
Turning point at $x=3$: $y=2(3)^2-12(3)+6=-12$, so $(3,-12)$.
$x$-intercepts: $2x^2-12x+6=0\Rightarrow x^2-6x+3=0
\Rightarrow x=3\pm\sqrt6$.

\item If roots are $r,s$, then $(r-s)^2=(r+s)^2-4rs=(p+1)^2-4p=(p-1)^2=9$.
So $p=4$ or $p=-2$.

\item Vertex at $(5,8)$ gives $y=a(x-5)^2+8$.
Using $y(1)=0$: $16a+8=0\Rightarrow a=-\dfrac12$.
So $y=-\dfrac12(x-5)^2+8$.
$y(7)=6$; the firework does not clear the wall (it meets it exactly at $6$ m).
For $y=6$: $-\dfrac12(x-5)^2+8=6\Rightarrow (x-5)^2=4\Rightarrow x=3 \text{ or } 7$.

\item $y(0)=\dfrac{10000}{500}+6=26$, so each tower is $26$ m high.
$y(40)=\dfrac{3600}{500}+6=\dfrac{66}{5}$.
For a $12$ m truck:
$12=\dfrac{1}{500}(x-100)^2+6\Rightarrow (x-100)^2=3000$,
so $x=100\pm 10\sqrt{30}$ (approximately $45.23$ or $154.77$).

\item Fencing gives $L+4x=200\Rightarrow L=200-4x$.
So $A=xL=x(200-4x)=200x-4x^2$.
Maximum at $x=\dfrac{200}{8}=25$ and then $L=100$.
Maximum area $=25\times 100=2500\text{ m}^2$.

\item Vertex at $x=\dfrac{-180}{2(-2)}=45$.
Maximum profit $P(45)=-2(45)^2+180(45)-2000=2050$.
Break-even: $-2x^2+180x-2000=0\Rightarrow x=45\pm 5\sqrt{41}$.
For $P(x)=1600$:
$-2x^2+180x-2000=1600\Rightarrow x^2-90x+1800=0
\Rightarrow x=30 \text{ or } 60$.

\item $c=2$.
From $(10,12)$: $100a+10b+2=12\Rightarrow 10a+b=1$.
From $(30,0)$: $900a+30b+2=0\Rightarrow 900a+30b=-2$.
Solving gives $a=-\dfrac{4}{75}$ and $b=\dfrac{23}{15}$.
So $h(x)=-\dfrac{4}{75}x^2+\dfrac{23}{15}x+2$.
Maximum at $x=-\dfrac{b}{2a}=\dfrac{115}{8}$ with height
$h\!\left(\dfrac{115}{8}\right)=\dfrac{625}{48}$.
At $x=12$, $h(12)=\dfrac{318}{25}<13$, so he does not clear the barrier.

\end{enumerate}

\end{multicols}

\end{document}
