\documentclass[12pt]{article}

\usepackage[a4paper,margin=1in]{geometry}
\usepackage{amsmath,amssymb}
\usepackage{enumitem}
\usepackage{multicol}

\setlist[enumerate]{leftmargin=*, itemsep=0.6em}

\begin{document}

\begin{center}
{\Large \textbf{Additional Maths Quadratic Chapter 1}}\\[4pt]
\end{center}

\noindent\textbf{Instructions.} Answer all questions. Where a sketch is requested, sketch the graph clearly and label all key features (intercepts, turning point, and line of symmetry where appropriate).

\vspace{1em}

% =========================
% QUESTIONS
% =========================

\section*{Section 1: Completing the Square}
\begin{enumerate}[label=\textbf{Q\arabic*.}]
\item Express $x^2+\dfrac{5}{2}x-7$ in the form $(x+a)^2+b$.
\item Express $5x^2-20x+7$ in the form $a(x+b)^2+c$.
\item Express $\dfrac12x^2-3x+4$ in the form $a(x+b)^2+c$.
\item Express $-\dfrac23x^2+4x-5$ in the form $a(x+b)^2+c$.
\end{enumerate}

% =========================
\section*{Section 2: Sketching Quadratic Graphs}
\begin{enumerate}[label=\textbf{Q\arabic*.}]
\item
\begin{enumerate}[label=(\roman*)]
\item Express $y=3x^2-12x+7$ in the form $a(x-h)^2+k$.
\item Hence, sketch the graph and state the turning point, the $x$-intercepts (if any), the $y$-intercept, and the line of symmetry.
\end{enumerate}

\item
\begin{enumerate}[label=(\roman*)]
\item Express $y=-2x^2+8x+1$ in the form $a(x-h)^2+k$.
\item Hence, sketch the graph and state the turning point, the $x$-intercepts (if any), the $y$-intercept, and the line of symmetry.
\end{enumerate}

\item Sketch the graph of $y=3(2x-1)(x+2)$ and state the $x$-intercepts, the $y$-intercept, the turning point, and the line of symmetry.

\item Sketch the graph of $y=(x+3)(5-2x)$ and state the $x$-intercepts, the $y$-intercept, the turning point, and the line of symmetry.
\end{enumerate}

% =========================
\section*{Section 3: Maximum/Minimum Values}
\begin{enumerate}[label=\textbf{Q\arabic*.}]
\item Given $f(x)=2x^2-8x+11$, find the smallest possible value of $f(x)$ and state the value of $x$ at which it occurs.

\item Explain why $x^2+6x+10$ is always positive for all real $x$, and state its minimum value.

\item Explain why $-4x^2-4x-2$ is always negative for all real $x$, and state its maximum value.

\item Find the maximum value of $y=-2x^2+7x-3$ and the value of $x$ at which it occurs.
\end{enumerate}

% =========================
\section*{Section 4: Applications of Quadratic Functions}
\begin{enumerate}[label=\textbf{Q\arabic*.}]
\item A cable between two towers has height above the roadway
\[
y=\frac{1}{900}(x-150)^2+9,\qquad 0\le x\le 300,
\]
where $x$ (m) is the horizontal distance from the left tower.
\begin{enumerate}[label=(\roman*)]
\item Find the height of each tower above the roadway.
\item Find the distance between the towers.
\item Find the possible distances from the left tower where the cable is $20$ m above the roadway.
\end{enumerate}

\item A bridge cable is supported by vertical wires. The two end wires are $35$ m long and are $180$ m apart. The shortest wire, at the midpoint, is $8$ m long.
Let $x$ be the horizontal distance (m) from the left end wire and $y$ the wire length (m).
\begin{enumerate}[label=(\roman*)]
\item Find $y$ in the form $y=a(x-h)^2+k$.
\item Find the wire length when $x=50$.
\end{enumerate}

\item A flare is launched from a platform $2$ m above ground. Its height (m) is modelled by
\[
y=-\frac1{25}x^2+\frac85x+2,
\]
where $x$ is the horizontal distance (m) from the launch point.
\begin{enumerate}[label=(\roman*)]
\item Find the maximum height and the horizontal distance at which it occurs.
\item Find the horizontal distance from the launch point when the flare hits the ground.
\item A mast at $x=30$ m is $15$ m tall. Determine whether the flare clears the mast.
\end{enumerate}

\item Two shells are fired from ground level at the origin. Their paths are modelled (in metres) by
\[
\text{Shell 1: } y=-\frac1{200}x^2+3x,
\qquad
\text{Shell 2: } y=-\frac1{500}(x-350)^2+245.
\]
\begin{enumerate}[label=(\roman*)]
\item Assuming Shell 1 hits an enemy frigate at sea level, how far from the origin is the frigate?
\item Find the maximum height of Shell 1.
\item Determine whether Shell 2 will hit the same frigate. Justify your answer.
\item What is the horizontal distance from the origin to where Shell 2 lands?
\end{enumerate}
\end{enumerate}

% =========================
% SOLUTIONS
% =========================
\newpage
\begin{center}
{\Large \textbf{Solutions}}\\[4pt]
\end{center}

\begin{multicols}{2}
\footnotesize

\section*{Section 1}
\begin{enumerate}[label=\textbf{Q\arabic*.}]
\item $\left(x+\dfrac54\right)^2-\dfrac{137}{16}$.
\item $5(x-2)^2-13$.
\item $\dfrac12(x-3)^2-\dfrac12$.
\item $-\dfrac23(x-3)^2+1$.
\end{enumerate}

\section*{Section 2}
\begin{enumerate}[label=\textbf{Q\arabic*.}]
\item $y=3(x-2)^2-5$.
Turning point $(2,-5)$, line of symmetry $x=2$, $y$-intercept $(0,7)$,
$x$-intercepts $\left(2\pm\dfrac{\sqrt{15}}{3},0\right)$.

\item $y=-2(x-2)^2+9$.
Turning point $(2,9)$, line of symmetry $x=2$, $y$-intercept $(0,1)$,
$x$-intercepts $\left(2\pm\dfrac{3\sqrt2}{2},0\right)$.

\item $x$-intercepts $(-2,0),\left(\dfrac12,0\right)$;
$y$-intercept $(0,-6)$; line of symmetry $x=-\dfrac34$;
turning point $\left(-\dfrac34,-\dfrac{75}{8}\right)$ (minimum).

\item $x$-intercepts $(-3,0),\left(\dfrac52,0\right)$;
$y$-intercept $(0,15)$; line of symmetry $x=-\dfrac14$;
turning point $\left(-\dfrac14,\dfrac{121}{8}\right)$ (maximum).
\end{enumerate}

\section*{Section 3}
\begin{enumerate}[label=\textbf{Q\arabic*.}]
\item $f(x)=2(x-2)^2+3$, so minimum value is $3$ at $x=2$.

\item $(x+3)^2+1\ge 1>0$, so it is always positive. Minimum value is $1$.

\item $-4\left(x+\dfrac12\right)^2-1\le -1<0$, so it is always negative. Maximum value is $-1$ at $x=-\dfrac12$.

\item $y=-2\left(x-\dfrac74\right)^2+\dfrac{25}{8}$, so maximum value is $\dfrac{25}{8}$ at $x=\dfrac74$.
\end{enumerate}

\section*{Section 4}
\begin{enumerate}[label=\textbf{Q\arabic*.}]
\item
\begin{enumerate}[label=(\roman*), leftmargin=*]
\item $y(0)=34$ and $y(300)=34$, so each tower is $34$ m high.
\item Distance between towers: $300$ m.
\item $20=\dfrac{1}{900}(x-150)^2+9\Rightarrow (x-150)^2=9900
\Rightarrow x=150\pm 30\sqrt{11}\approx 50.5\text{ m or }249.5$ m.
\end{enumerate}

\item
\begin{enumerate}[label=(\roman*), leftmargin=*]
\item Vertex $(90,8)$: $y=a(x-90)^2+8$.
Using $(0,35)$ gives $35=8100a+8\Rightarrow a=\dfrac{1}{300}$,
so $y=\dfrac{1}{300}(x-90)^2+8$.
\item $y(50)=\dfrac{1}{300}(50-90)^2+8=\dfrac{40}{3}$ m.
\end{enumerate}

\item
\begin{enumerate}[label=(\roman*), leftmargin=*]
\item Vertex at $x=20$, maximum height $y(20)=18$ m.
\item $0=-\dfrac1{25}x^2+\dfrac85x+2
\Rightarrow x=20\pm 15\sqrt2$; physical root $x=20+15\sqrt2\approx 41.2$ m.
\item $y(30)=14<15$, so it does not clear the mast.
\end{enumerate}

\item
\begin{enumerate}[label=(\roman*), leftmargin=*]
\item $0=-\dfrac1{200}x^2+3x\Rightarrow x=600$ m (non-zero root).
\item Maximum at $x=300$: $y(300)=450$ m.
\item At $x=600$, Shell 2 has $y=-\dfrac1{500}(250)^2+245=120>0$, so it is still airborne and does not hit the frigate.
\item Shell 2 lands when $0=-\dfrac1{500}(x-350)^2+245\Rightarrow (x-350)^2=350^2\Rightarrow x=700$ m.
\end{enumerate}
\end{enumerate}

\end{multicols}

\end{document}
