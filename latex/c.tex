\documentclass[12pt]{article}

\usepackage[a4paper,margin=1in]{geometry}
\usepackage{amsmath,amssymb}
\usepackage{enumitem}
\usepackage{multicol}

\setlist[enumerate]{leftmargin=*, itemsep=0.6em}

\begin{document}

\begin{center}
{\Large \textbf{Quadratic Functions --- Assignment}}\\[4pt]
\end{center}

\noindent\textbf{Instructions.} Answer all questions. Where a sketch is requested, sketch the graph clearly and label all key features (intercepts, turning point, and line of symmetry where appropriate). No sketches are provided in the solutions.

\vspace{1em}

% =========================
\section*{Section 1: Completing the Square (Part 1a--1b)}
\begin{enumerate}[label=\textbf{Q\arabic*.}]
\item Express $x^2-14x+10$ in the form $(x+a)^2+b$.

\item Express $x^2+3x+9$ in the form $(x+a)^2+b$.

\item Express $x^2+\dfrac{5}{2}x-7$ in the form $(x+a)^2+b$.

\item Express $x^2+8x+1$ in the form $(x+a)^2+b$.

\item Express $x^2-9x+2$ in the form $(x+a)^2+b$.

\item Express $5x^2-20x+7$ in the form $a(x+b)^2+c$.

\item Express $-3x^2+6x+1$ in the form $a(x+b)^2+c$.

\item Express $\dfrac12x^2-3x+4$ in the form $a(x+b)^2+c$.

\item Express $6x^2+5x-1$ in the form $a(x+b)^2+c$.

\item Express $-\dfrac23x^2+4x-5$ in the form $a(x+b)^2+c$.
\end{enumerate}

% =========================
\section*{Section 2: Sketching Quadratic Graphs (Part 2a--2b)}
\begin{enumerate}[label=\textbf{Q\arabic*.}, start=11]

\item
\begin{enumerate}[label=(\roman*)]
\item Express $y=3x^2-12x+7$ in the form $a(x-h)^2+k$.
\item Hence, sketch the graph and state the turning point, the $x$-intercepts (if any), the $y$-intercept, and the line of symmetry.
\end{enumerate}

\item
\begin{enumerate}[label=(\roman*)]
\item Express $y=-2x^2+8x+1$ in the form $a(x-h)^2+k$.
\item Hence, sketch the graph and state the turning point, the $x$-intercepts (if any), the $y$-intercept, and the line of symmetry.
\end{enumerate}

\item
\begin{enumerate}[label=(\roman*)]
\item Express $y=x^2+4x+6$ in the form $a(x-h)^2+k$.
\item Hence, sketch the graph and state the number of $x$-intercepts.
\end{enumerate}

\item
\begin{enumerate}[label=(\roman*)]
\item Express $f(x)=-3x^2-6x+12$ in the form $a(x-h)^2+k$.
\item Hence, state the maximum value of $f(x)$ and the value of $x$ at which it occurs.
\item Sketch the graph and label the key features.
\end{enumerate}

\item
\begin{enumerate}[label=(\roman*)]
\item Express $y=2x^2-4x-6$ in the form $a(x-h)^2+k$.
\item Hence, solve $2x^2-4x-6\ge 0$.
\item Sketch the graph and label the key features.
\end{enumerate}

\item Sketch the graph of $y=(x-5)(x+1)$ and state the $x$-intercepts, the $y$-intercept, the turning point, and the line of symmetry.

\item Sketch the graph of $y=-2(x-3)(x+4)$ and state the $x$-intercepts, the $y$-intercept, the turning point, and the line of symmetry.

\item Sketch the graph of $y=3(2x-1)(x+2)$ and state the $x$-intercepts, the $y$-intercept, the turning point, and the line of symmetry.

\item Sketch the graph of $y=(x-2)(x-2)$ and state the $x$-intercepts (including multiplicity), the $y$-intercept, the turning point, and the line of symmetry.

\item Sketch the graph of $y=(x+3)(5-2x)$ and state the $x$-intercepts, the $y$-intercept, the turning point, and the line of symmetry.

\end{enumerate}

% =========================
\section*{Section 3: Maximum/Minimum Values and Inequalities (Part 3)}
\begin{enumerate}[label=\textbf{Q\arabic*.}, start=21]

\item Given $f(x)=2x^2-8x+11$, find the smallest possible value of $f(x)$ and state the value of $x$ at which it occurs.

\item Given $y=-3x^2+12x-7$, find the range of possible values of $y$.

\item Show that for all real $x$,
\[
5x^2-20x+18\ge -2,
\]
and state when equality holds.

\item Show that for all real $x$,
\[
-2x^2+4x-9\le -7,
\]
and state when equality holds.

\item Explain why $x^2+6x+10$ is always positive for all real $x$, and state its minimum value.

\item Explain why $-4x^2-4x-2$ is always negative for all real $x$, and state its maximum value.

\item Find the maximum value of $y=-2x^2+7x-3$ and the value of $x$ at which it occurs.

\item Given $y=3x^2-12x+k$, the minimum value of $y$ is $5$. Find $k$, and hence state the range of $y$.

\item Find the least value of $x^2-4x+7$. Hence find the range of
\[
\frac{1}{x^2-4x+7}.
\]

\item Let $g(x)=2x^2-5x+4$.
\begin{enumerate}[label=(\roman*)]
\item Find the range of $g(x)$.
\item Hence solve $2x^2-5x+4\le 1$.
\end{enumerate}

\end{enumerate}

% =========================
\section*{Section 4: Applications of Quadratic Functions (A Maths Chap 1 Section 4)}
\begin{enumerate}[label=\textbf{Q\arabic*.}, start=31]

\item A cable between two towers has height above the roadway
\[
y=\frac{1}{900}(x-150)^2+9,\qquad 0\le x\le 300,
\]
where $x$ (m) is the horizontal distance from the left tower.
\begin{enumerate}[label=(\roman*)]
\item Find the height of each tower above the roadway.
\item Find the distance between the towers.
\item Find the possible distances from the left tower where the cable is $20$ m above the roadway.
\end{enumerate}

\item Two towers are $18$ m above the roadway and $90$ m apart. The cable between them just touches the roadway halfway between the towers.
Let $x$ be the horizontal distance (m) from the left tower and $y$ the height (m) above the roadway.
\begin{enumerate}[label=(\roman*)]
\item Find a quadratic function for $y$ in terms of $x$.
\item Find the cable's height above the roadway at a point $15$ m from the left tower.
\end{enumerate}

\item A bridge cable is supported by vertical wires. The two end wires are $35$ m long and are $180$ m apart. The shortest wire, at the midpoint, is $8$ m long.
Let $x$ be the horizontal distance (m) from the left end wire and $y$ the wire length (m).
\begin{enumerate}[label=(\roman*)]
\item Find $y$ in the form $y=a(x-h)^2+k$.
\item Find the wire length when $x=50$.
\end{enumerate}

\item A skateboard ramp is modelled by
\[
y=0.5x^2-4x+9,
\]
where $x$ is horizontal distance (m) and $y$ is height above the ground (m).
\begin{enumerate}[label=(\roman*)]
\item Find the $y$-intercept and the turning point.
\item Find the minimum height of the ramp above the ground.
\item For safety, the ramp height is limited to $3$ m. Find the maximum possible width of the curved part (the distance between the two points where $y=3$).
\end{enumerate}

\item A flare is launched from a platform $2$ m above ground. Its height (m) is modelled by
\[
y=-\frac1{25}x^2+\frac85x+2,
\]
where $x$ is the horizontal distance (m).
\begin{enumerate}[label=(\roman*)]
\item Find the maximum height and where it occurs.
\item Find the horizontal distance from the launch point when the flare hits the ground.
\item A mast at $x=30$ m is $15$ m tall. Determine whether the flare clears the mast.
\end{enumerate}

\item Two shells are fired from ground level at the origin. Their paths are modelled (in metres) by
\[
\text{Shell 1: } y=-\frac1{200}x^2+3x,
\qquad
\text{Shell 2: } y=-\frac1{500}(x-350)^2+245.
\]
\begin{enumerate}[label=(\roman*)]
\item Assuming Shell 1 hits an enemy frigate at sea level, how far from the origin is the frigate?
\item Find the maximum height of Shell 1.
\item Determine whether Shell 2 will hit the same frigate. Justify your answer.
\item What is the horizontal distance from the origin to where Shell 2 lands?
\end{enumerate}

\item A wire of length $60$ cm is bent into a rectangle. The width is $x$ cm.
\begin{enumerate}[label=(\roman*)]
\item Express the area $A$ in terms of $x$.
\item Find the maximum possible area.
\item Show that this maximum occurs only when the rectangle is a square.
\end{enumerate}

\item The height (m) of a ball at time $t$ seconds is
\[
y=-3t^2+18t+c,
\]
where $c$ is a constant.
\begin{enumerate}[label=(\roman*)]
\item Use the discriminant to find the range of $c$ if the ball does not reach $40$ m.
\item If $c=10$, write $y$ in the form $a(t-h)^2+k$ and state the maximum point.
\item Explain how your result in (ii) is consistent with the condition found in (i).
\end{enumerate}

\item Three water jets are modelled by
\[
A:\ y=-0.25x^2+4,\quad
B:\ y=-0.16x^2+3.2,\quad
C:\ y=-0.10x^2+3.6,
\]
where $x$ is horizontal distance (m) and $y$ is height (m).
\begin{enumerate}[label=(\roman*)]
\item Which jet sends water the farthest horizontally (largest positive $x$-intercept)?
\item Which jet sends water the highest?
\item Which jet has the steepest curve (largest $|a|$ value)? Give a brief reason.
\item A fourth jet is modelled by $D:\ y=-0.2x^2+2x+k$.
Find the range of $k$ such that the maximum height is at least $10$ m.
\end{enumerate}

\item A soccer ball is kicked from ground level. Its height is
\[
h(t)=22t-5t^2,
\]
where $t$ is in seconds and $h$ in metres.
\begin{enumerate}[label=(\roman*)]
\item Express $h(t)$ in the form $a(t-p)(t-q)$.
\item Find the maximum height and the time when it occurs.
\item Find the time when the ball returns to the ground.
\item Find the time when the ball is at height $18$ m on the way down.
\end{enumerate}

\end{enumerate}

% =========================
\newpage
\section*{Solutions}

\begin{multicols}{2}
\footnotesize

\begin{enumerate}[label=\textbf{Q\arabic*.}, leftmargin=*]

\item $(x-7)^2-39$.

\item $\left(x+\dfrac32\right)^2+\dfrac{27}{4}$.

\item $\left(x+\dfrac54\right)^2-\dfrac{137}{16}$.

\item $(x+4)^2-15$.

\item $\left(x-\dfrac92\right)^2-\dfrac{73}{4}$.

\item $5(x-2)^2-13$.

\item $-3(x-1)^2+4$.

\item $\dfrac12(x-3)^2-\dfrac12$.

\item $6\left(x+\dfrac{5}{12}\right)^2-\dfrac{49}{24}$.

\item $-\dfrac23(x-3)^2+1$.

\item $y=3(x-2)^2-5$.
Turning point $(2,-5)$, line of symmetry $x=2$, $y$-intercept $(0,7)$,
$x$-intercepts $\left(2\pm\dfrac{\sqrt{15}}{3},0\right)$.

\item $y=-2(x-2)^2+9$.
Turning point $(2,9)$, line of symmetry $x=2$, $y$-intercept $(0,1)$,
$x$-intercepts $\left(2\pm\dfrac{3\sqrt2}{2},0\right)$.

\item $y=(x+2)^2+2$.
Turning point $(-2,2)$, line of symmetry $x=-2$, $y$-intercept $(0,6)$, no $x$-intercepts.

\item $f(x)=-3(x+1)^2+15$.
Maximum $15$ at $x=-1$.
Turning point $(-1,15)$, line of symmetry $x=-1$, $y$-intercept $(0,12)$,
$x$-intercepts $\left(-1\pm\sqrt5,0\right)$.

\item $y=2(x-1)^2-8$.
$x$-intercepts $(-1,0),(3,0)$; $y$-intercept $(0,-6)$.
Hence $2x^2-4x-6\ge 0$ for $x\le -1$ or $x\ge 3$.
Turning point $(1,-8)$, line of symmetry $x=1$.

\item $x$-intercepts $(-1,0),(5,0)$; $y$-intercept $(0,-5)$;
line of symmetry $x=2$; turning point $(2,-9)$ (minimum).

\item $x$-intercepts $(-4,0),(3,0)$; $y$-intercept $(0,24)$;
line of symmetry $x=-\dfrac12$; turning point $\left(-\dfrac12,\dfrac{49}{2}\right)$ (maximum).

\item $x$-intercepts $(-2,0),\left(\dfrac12,0\right)$; $y$-intercept $(0,-6)$;
line of symmetry $x=-\dfrac34$; turning point $\left(-\dfrac34,-\dfrac{75}{8}\right)$ (minimum).

\item $y=(x-2)^2$.
$x$-intercept $(2,0)$ (double root); $y$-intercept $(0,4)$;
line of symmetry $x=2$; turning point $(2,0)$ (minimum).

\item $x$-intercepts $(-3,0),\left(\dfrac52,0\right)$; $y$-intercept $(0,15)$;
line of symmetry $x=-\dfrac14$; turning point $\left(-\dfrac14,\dfrac{121}{8}\right)$ (maximum).

\item $f(x)=2(x-2)^2+3$.
Minimum value $3$ at $x=2$.

\item $y=-3(x-2)^2+5$.
Range: $y\le 5$ (maximum $5$ at $x=2$).

\item $5x^2-20x+18=5(x-2)^2-2\ge -2$; equality at $x=2$.

\item $-2x^2+4x-9=-2(x-1)^2-7\le -7$; equality at $x=1$.

\item $x^2+6x+10=(x+3)^2+1\ge 1>0$; minimum value $1$.

\item $-4x^2-4x-2=-4\left(x+\dfrac12\right)^2-1<0$; maximum value $-1$ at $x=-\dfrac12$.

\item $y=-2\left(x-\dfrac74\right)^2+\dfrac{25}{8}$.
Maximum $\dfrac{25}{8}$ at $x=\dfrac74$.

\item $y=3(x-2)^2+(k-12)$.
Minimum $k-12=5\Rightarrow k=17$.
Range: $y\ge 5$.

\item $x^2-4x+7=(x-2)^2+3\ge 3$.
Least value $3$.
So $0<\dfrac{1}{x^2-4x+7}\le \dfrac13$.

\item $g(x)=2\left(x-\dfrac54\right)^2+\dfrac78$.
\begin{enumerate}[label=(\roman*), leftmargin=*]
\item Range: $g(x)\ge \dfrac78$.
\item $2\left(x-\dfrac54\right)^2+\dfrac78\le 1
\Rightarrow \left(x-\dfrac54\right)^2\le \dfrac{1}{16}
\Rightarrow 1\le x\le \dfrac32$.
\end{enumerate}

\item
\begin{enumerate}[label=(\roman*), leftmargin=*]
\item $y(0)=34$ and $y(300)=34$, so each tower is $34$ m high.
\item Distance between towers: $300$ m.
\item $20=\dfrac{1}{900}(x-150)^2+9\Rightarrow (x-150)^2=9900
\Rightarrow x=150\pm 30\sqrt{11}\approx 50.5\text{ or }249.5$.
\end{enumerate}

\item
\begin{enumerate}[label=(\roman*), leftmargin=*]
\item Vertex $(45,0)$: $y=a(x-45)^2$.
Using $(0,18)$ gives $18=a(45)^2\Rightarrow a=\dfrac{2}{225}$,
so $y=\dfrac{2}{225}(x-45)^2$.
\item $y(15)=\dfrac{2}{225}(15-45)^2=8$ m.
\end{enumerate}

\item
\begin{enumerate}[label=(\roman*), leftmargin=*]
\item Vertex $(90,8)$: $y=a(x-90)^2+8$.
Using $(0,35)$ gives $35=8100a+8\Rightarrow a=\dfrac{1}{300}$,
so $y=\dfrac{1}{300}(x-90)^2+8$.
\item $y(50)=\dfrac{1}{300}(50-90)^2+8=\dfrac{40}{3}$ m.
\end{enumerate}

\item
\begin{enumerate}[label=(\roman*), leftmargin=*]
\item $y$-intercept $(0,9)$; turning point at $x=4$, $y(4)=1$, so $(4,1)$.
\item Minimum height: $1$ m.
\item $0.5x^2-4x+9=3\Rightarrow x^2-8x+12=0\Rightarrow x=2,6$.
Width $=6-2=4$ m.
\end{enumerate}

\item
\begin{enumerate}[label=(\roman*), leftmargin=*]
\item Vertex at $x=20$, maximum height $y(20)=18$ m.
\item $0=-\dfrac1{25}x^2+\dfrac85x+2
\Rightarrow x=20\pm 15\sqrt2$; physical root $x=20+15\sqrt2\approx 41.2$ m.
\item $y(30)=14<15$, so it does not clear the mast.
\end{enumerate}

\item
\begin{enumerate}[label=(\roman*), leftmargin=*]
\item $0=-\dfrac1{200}x^2+3x\Rightarrow x=600$ m (non-zero root).
\item Maximum at $x=300$: $y(300)=450$ m.
\item At $x=600$, Shell 2 has $y=-\dfrac1{500}(250)^2+245=120>0$, so it does not hit.
\item Shell 2 lands when $0=-\dfrac1{500}(x-350)^2+245\Rightarrow (x-350)^2=350^2\Rightarrow x=700$ m.
\end{enumerate}

\item
\begin{enumerate}[label=(\roman*), leftmargin=*]
\item Perimeter $60$: length $=30-x$. So $A=x(30-x)=-x^2+30x$.
\item Maximum at $x=15$: $A_{\max}=225\text{ cm}^2$.
\item When $x=15$, length $=15$, so the rectangle is a square; maximum is unique at the vertex.
\end{enumerate}

\item
\begin{enumerate}[label=(\roman*), leftmargin=*]
\item Not reach $40$ means $-3t^2+18t+(c-40)=0$ has no real root:
$\Delta=12c-156<0\Rightarrow c<13$.
\item If $c=10$: $y=-3(t-3)^2+37$, maximum point $(3,37)$.
\item Maximum height is $27+c$; requiring $27+c<40$ gives $c<13$ (consistent with (i)).
\end{enumerate}

\item
\begin{enumerate}[label=(\roman*), leftmargin=*]
\item Positive $x$-intercepts: $A:4$, $B:\sqrt{20}$, $C:6$; farthest is $C$.
\item Highest is $A$ (height $4$ at $x=0$).
\item Narrowest is $A$ (largest $|a|$).
\item $-0.2x^2+2x+k=-0.2(x-5)^2+(5+k)$, so max height $5+k\ge 10\Rightarrow k\ge 5$.
\end{enumerate}

\item
\begin{enumerate}[label=(\roman*), leftmargin=*]
\item $h(t)=22t-5t^2=-5t\left(t-\dfrac{22}{5}\right)$.
\item Vertex at $t=\dfrac{11}{5}$; maximum height $h\!\left(\dfrac{11}{5}\right)=\dfrac{121}{5}$ m.
\item Returns to ground at $t=\dfrac{22}{5}$ s.
\item $22t-5t^2=18\Rightarrow t=\dfrac{11\pm\sqrt{31}}{5}$; on the way down:
$t=\dfrac{11+\sqrt{31}}{5}\approx 3.31$ s.
\end{enumerate}

\end{enumerate}

\end{multicols}

\end{document}
